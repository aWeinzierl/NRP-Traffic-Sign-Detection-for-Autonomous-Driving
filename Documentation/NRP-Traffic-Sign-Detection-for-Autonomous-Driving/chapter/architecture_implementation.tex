
\section{Architecture and Implementation}

Description of the architecture of the experiment......

\subsection{Speed Limit Sign Section}

The traffic sign model in the experiment is the most important part of the environment as it directly influences the car’s decision making. For the simplicity, only speed limit signs are to be used because this sign can already lead to a visible effect (acceleration or deceleration) on the car. The traffic sign cognition function of the car can therefore be obviously proved. 



The technique used in the car to recognize the traffic sign is neural network. And the training dataset of the neural network is a fully generated dataset. As this dataset doesn’t include all types of speed limit signs, the model should include the right speed limit model (rather than a random one) so that the trained model in the car can detect and recognize the speed limit sign in the right way. In the experiment, speed limit signs for 100 km/h and 20 km/h are used.



A compatible model in NRP is defined mainly by three files: ``limit20.dae'', ``model.config'', and ``model.sdf''. The ``model.config'' file configures the author, model name, model version and description. The ``model.sdf defines'' the physical property of the model like geometry and collision property. The ``limit20.dae'' specifies the property of the speed limit sign in detail. For example, texture, material, and relative position of the traffic signs are defined in this file.



To speed up the model building, a reference model in ``.skp'' format is downloaded from the internet. To change it into a ``.dae'' format which is compatible with NRP, the 3D modeling software ``Sketchup'' in operating system Windows 10 (this software is not supported in Ubuntu) is used to convert the file to ``.dae'' format. The following operations are implemented in Ubuntu 16.04 operating system. 



After having the above files, the scripts are edited according to the model physical property, model name, model directory, and reference pictures. A new ``.sdf'' file is created to configure the geometry, link, collision property with the use of the ``.dae'' file. 

After implementing the above steps, the model should be available in the NRP platform.





\begin{table}[htpb]
	\caption[Example table]{An example for a simple table.}\label{tab:sampleA}
	\centering
	\begin{tabular}{l l l l}
		A & B & C & D \\
		\cline{1-4}	
		1 & 2 & 1 & 2 \\
		2 & 3 & 2 & 3 \\
	\end{tabular}
\end{table}



